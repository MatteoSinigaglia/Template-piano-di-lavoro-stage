%----------------------------------------------------------------------------------------
%	OBJECTIVES
%----------------------------------------------------------------------------------------
\section*{Obiettivi}
\subsection*{Notazione}
Si farà riferimento ai requisiti secondo le seguenti notazioni:
\begin{itemize}
	\item \textit{O} per i requisiti obbligatori, vincolanti in quanto obiettivo primario richiesto dal committente;
	\item \textit{D} per i requisiti desiderabili, non vincolanti o strettamente necessari,
		  ma dal riconoscibile valore aggiunto;
	\item \textit{F} per i requisiti facoltativi, rappresentanti valore aggiunto non strettamente 
		  competitivo.
\end{itemize}

Le sigle precedentemente indicate saranno seguite da una coppia sequenziale di numeri, identificativo del requisito.

\subsection*{Obiettivi fissati}
Si prevede lo svolgimento dei seguenti obiettivi:
\begin{itemize}
	\item Obbligatori
	\begin{itemize}
		\item O01: Acquisizione competenze sulle tematiche sopra descritte;
		\item O02: Capacità di raggiungere gli obiettivi richiesti in autonomia seguendo il cronoprogramma;
		\item O03: Portare a termine le implementazioni previste con una percentuale di superamento pari al 80\%.
	\end{itemize}
	
	\item Desiderabili 
	\begin{itemize}
		\item D01: Portare a termine le implementazioni previste con una percentuale di superamento pari al 100\%.
	\end{itemize}
	
	\item Facoltativi
	\begin{itemize}
		\item F01: Riuscire a studiare e prevedere nell’implementazione la gestione del JWT Token durante la navigazione all’interno della web app.
	\end{itemize} 
\end{itemize}

\newpage