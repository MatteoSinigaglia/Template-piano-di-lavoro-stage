%----------------------------------------------------------------------------------------
%   USEFUL COMMANDS
%----------------------------------------------------------------------------------------

\newcommand{\dipartimento}{Dipartimento di Matematica ``Tullio Levi-Civita''}

%----------------------------------------------------------------------------------------
% 	USER DATA
%----------------------------------------------------------------------------------------

% Data di approvazione del piano da parte del tutor interno; nel formato GG Mese AAAA
% compilare inserendo al posto di GG 2 cifre per il giorno, e al posto di 
% AAAA 4 cifre per l'anno
\newcommand{\dataApprovazione}{Data}

% Dati dello Studente
\newcommand{\nomeStudente}{Matteo}
\newcommand{\cognomeStudente}{Sinigaglia}
\newcommand{\matricolaStudente}{1193412}
\newcommand{\emailStudente}{matteo.sinigaglia.6@studenti.unipd.it}
\newcommand{\telStudente}{+ 39 345 22 53 700}

% Dati del Tutor Aziendale
\newcommand{\nomeTutorAziendale}{Nicola}
\newcommand{\cognomeTutorAziendale}{Pastore}
\newcommand{\emailTutorAziendale}{nicola.pastore@synclab.it}
\newcommand{\telTutorAziendale}{+ 39 333 13 68 500}
\newcommand{\ruoloTutorAziendale}{Sviluppatore AngularJS}

% Dati dell'Azienda
\newcommand{\ragioneSocAzienda}{Sync Lab S.r.l.}
\newcommand{\indirizzoAzienda}{Galleria Spagna, 28 (PD)}
\newcommand{\sitoAzienda}{www.synclab.it}
\newcommand{\emailAzienda}{info@synclab.it}
\newcommand{\partitaIVAAzienda}{P.IVA 07952560634}

% Dati del Tutor Interno (Docente)
\newcommand{\titoloTutorInterno}{Prof.}
\newcommand{\nomeTutorInterno}{NomeDocente}
\newcommand{\cognomeTutorInterno}{CognomeDocente}

\newcommand{\prospettoSettimanale}{
     % Personalizzare indicando in lista, i vari task settimana per settimana
     % sostituire a XX il totale ore della settimana
    \begin{itemize}
        \item \textbf{Prima Settimana (40 ore)}
        \begin{itemize}
            \item Incontro con persone coinvolte nel progetto per discutere i requisiti e le richieste relativamente 
            al sistema da sviluppare;
            \item Presentazione strumenti di lavoro per la condivisione del materiale di studio e per la gestione 
            dell'avanzamento;
            \item Condivisione scaletta di argomenti;
            \item Ripasso del linguaggio Java SE;
            \item Ripasso concetti Web (Servlet, servizi Rest, Json ecc.).
        \end{itemize}
        \item \textbf{Seconda Settimana (40 ore)} 
        \begin{itemize}
            \item Studio principi generali di Spring Core (IOC, Dependency Injection);
            \item Studio SpringBoot;
            \item Studio Spring Data/DataRest.
        \end{itemize}
        \item \textbf{Terza Settimana (40 ore)} 
        \begin{itemize}
            \item Ripasso linguaggio Javascript;
            \item Studio del linguaggio TypeScript.
        \end{itemize}
        \item \textbf{Quarta Settimana (40 ore)} 
        \begin{itemize}
            \item Studio piattaforma NodeJS e AngularCLI;
            \item Studio framework Angular.
        \end{itemize}
        \item \textbf{Quinta Settimana (40 ore)} 
        \begin{itemize}
            \item Analisi e studio del progetto SportWill;
            \item Progettazione ed implementazione della nuova maschera di login.
        \end{itemize}
        \item \textbf{Sesta Settimana (40 ore)} 
        \begin{itemize}
            \item Progettazione ed implementazione nuova maschera "Gestione Uscite";
            \item Scrittura dei service (su front-end) di chiamata al back-end.
        \end{itemize}
        \item \textbf{Settima Settimana (40 ore)} 
        \begin{itemize}
            \item Termine implementazioni ed integrazione con l'applicativo.
        \end{itemize}
        \item \textbf{Ottava Settimana - Conclusione (40 ore)} 
        \begin{itemize}
            \item Termine integrazioni e collaudo finale.
        \end{itemize}
    \end{itemize}
}

% Indicare il totale complessivo (deve essere compreso tra le 300 e le 320 ore)
\newcommand{\totaleOre}{320}

\newcommand{\obiettiviObbligatori}{
	 \item \underline{\textit{O01}}: primo obiettivo;
	 \item \underline{\textit{O02}}: secondo obiettivo;
	 \item \underline{\textit{O03}}: terzo obiettivo;
	 
}

\newcommand{\obiettiviDesiderabili}{
	 \item \underline{\textit{D01}}: primo obiettivo;
	 \item \underline{\textit{D02}}: secondo obiettivo;
}

\newcommand{\obiettiviFacoltativi}{
	 \item \underline{\textit{F01}}: primo obiettivo;
	 \item \underline{\textit{F02}}: secondo obiettivo;
	 \item \underline{\textit{F03}}: terzo obiettivo;
}